%Part of/Parte di https://github.com/f-dinucci/appuntiMeccanicaFluidi/
%License/Licenza Creative Commons Attribution-ShareAlike 4.0 International (CC BY-SA 4.0) - attribution/attribuzione Francesco Di Nucci
%See also/Vedere anche https://creativecommons.org/licenses/by-sa/4.0/ and/e https://creativecommons.org/licenses/by-sa/4.0/legalcode
%
\section{Equazione dell'energia meccanica}
L'equazione di Bernoulli è l'equazione dell'energia \textit{meccanica}, non dell'energia in generale. 
È una equazione derivata da quella della quantità di moto, non è una relazione fondamentale e permette di avere informazioni finite su alcune grandezze. 

%SUBSECTION
\subsection{Forma debole}
Nella meccanica discreta si può ottenere l'equazione dell'energia meccanica moltiplicando scalarmente la legge di Newton per la velocità:
%
	\begin{equation*}
		\begin{gathered}
			\uline{F} = m \pdv[2]{x}{t}\\\left(\frac{v^2}{2} + \frac{p}{\rho} + gz \right) \rho \uline{v}
			\dv{\uline{x}}{t} \vdot \uline{F} = m \dv{\uline{x}}{t} \vdot \pdv[2]{x}{t}\\
			\text{se la forza deriva da un potenziale}\\
			\uline{F} = - \dv{U}{\uline{x}}\\
			- \dd{U}{t} = m \dv{t} \left( \frac{1}{2} \dv{x}{t} \right)\\
		\end{gathered}
	\end{equation*}
%

Nel caso continuo invece, dalle equazioni di Eulero (in caso quindi di fluido non viscoso):
%
	\begin{equation*}
		\begin{gathered}
			\pdv{\uline{v}}{t} + \uline{v} \div{\uline{v}} + \frac{1}{\rho} \grad{p}  = \uline{g}\\
		\end{gathered}
	\end{equation*}
%		
Moltiplicando per la velocità:
%
	\begin{equation*}
		\begin{gathered}
			\uline{v} \vdot \pdv{\uline{v}}{t} + \uline{v} \vdot \uline{v} \div{\uline{v}} + \frac{1}{\rho} \grad{p}  \vdot \uline{v} = \uline{g} \vdot \uline{v}\\
			\uline{v} \vdot \pdv{\uline{v}}{t} + \uline{v} \vdot \grad{\frac{v^2}{2}} + \frac{1}{\rho} \grad{p}  \vdot \uline{v} = \uline{g} \vdot \uline{v}\\
		\end{gathered}
	\end{equation*}
%
Vengono aggiunte alcune ipotesi.
In caso di moto stazionario $\pdv{\uline{v}}{t} = 0$.
Nel caso di accelerazione di gravità costante, questa sicuramente deriva da un potenziale (preso l'asse z verticale) $\uline{g} = \grad{(-gz)}$.
Aggiungendo l'ipotesi di fluido incomprimibile ($\rho = costante$), si ottiene:
%
	\begin{equation*}
		\uline{v} \vdot \grad{\left( \frac{v^2}{2}\right)} + \frac{p}{\rho} \grad{p} \vdot \uline{v} + \grad{(- gz)} \vdot \uline{v} = 0
	\end{equation*}
%
Si ottiene quindi la \textbf{legge di Bernoulli in forma debole}:
%
	\begin{equation*}
		\uline{v} \vdot \grad{\left( \frac{v^2}{2} + \frac{p}{\rho} + gz \right)}  = 0
	\end{equation*}
%
Ricordando che una linea di corrente è una linea che punto per punto è parallela al vettore velocità, dato che la componente lungo v della quantità in parentesi è nulla, questa quantità è costante lungo una linea di corrente.
Il termine tra parentesi viene detto trinomio di Bernoulli ed è costante lungo una linea di corrente:
%
	\begin{equation*}
		\left( \frac{v^2}{2} + \frac{p}{\rho} + gz \right)
	\end{equation*}
%

%SUBSECTION
\subsection{Forma forte}
Esiste anche la ``forma forte'' dell'equazione dell'energia meccanica.
Partendo nuovamente dall'equazione di Eulero:
%
	\begin{equation*}
		\pdv{\uline{v}}{t} + \uline{v} \div{\uline{v}} + \frac{1}{\rho} \grad{p}  = \uline{g}
	\end{equation*}
 %
 
Per dei vettori algebrici vale che (e si può fare anche se uno degli operatori è nabla, basta stare attenti all'ordine):
%
	\begin{equation*}
		\begin{gathered}
			\uline{a} \cross (\uline{b} \cross \uline{c}) = (\uline{a} \vdot \uline{c}) \uline{b} - (\uline{a} \vdot \uline{b}) \uline{c}\\
			\text{se c è nabla}\\
			\uline{v} \vdot \grad{\uline{v}} \quad \text{come} \quad (\uline{a} \vdot \uline{c}) \uline{b}\\
			- \uline{v} \cross (\curl{\uline{v}}) = \uline{v} \vdot \grad{\uline{v}} - (\grad{\uline{v}}) \vdot \uline{v}\\
			\uline{v} \vdot \grad{\uline{v}} = (\curl{\uline{v}}) \cross \uline{v} + \grad{\uline{v}} \vdot \uline{v}
		\end{gathered}
	\end{equation*}
%
Si introduce quindi la vorticità come rotore della velocità:
%
	\begin{equation*}
 		\uline{\omega} = \curl{\uline{v}}
	\end{equation*}
%
Ricordando poi che $\grad{\uline{v}} \vdot \uline{v} = \grad{ ( \frac{v^2}{2} ) }$ si arriva a:
%
	\begin{equation*}
		\uline{v} \vdot \grad{\uline{v}} = \uline{\omega} \cross \uline{v} + \grad{(\frac{v^2}{2})}
	\end{equation*}
%
Sostituendo nell'equazione iniziale e ripetendo le considerazioni precedenti sulla densità e gravità costanti, si arriva a:
%
	\begin{equation*}
		\pdv{\uline{v}}{t} + \uline{\omega} \cross \uline{v} + \grad{\left( \frac{v^2}{2} + \frac{p}{\rho} + gz \right)} = 0
	\end{equation*}
%
Notare che nel caso stazionario darebbe lo stesso risultato della forma debole, dato che si avrebbero $\pdv{\uline{v}}{t} = 0$ e $\uline{v} \vdot (\uline{\omega} \vdot \uline{v}) = 0$.

L'interpretazione del risultato ottenuto è che il trinomio di Bernoulli è costante anche lungo linee parallele punto per punto al vettore vorticità, dette linee vorticose, dato che:
%
	\begin{equation*}
		 \uline{\omega} \vdot (\uline{\omega} \cross \uline{v}) = 0
	\end{equation*}
%

Si può introdurre una ulteriore ipotesi, che $\uline{\omega} = 0$, in tal caso il moto è detto irrotazionale e $\uline{\omega} = \curl{\uline{v}} = 0$.
Si ottiene \textbf{il trinomio di Bernoulli in forma forte}:
%
	\begin{equation*}
		\grad{\left( \frac{v^2}{2} + \frac{p}{\rho} + gz \right)} = 0
	\end{equation*}
%
Sono tre condizioni, il trinomio di Bernoulli è costante in \textit{tutto} lo spazio in cui la vorticità è nulla, per questo è detta forma forte.

%SUBSECTION
\subsection{Problema instazionario}
La forma forte è valida anche in caso di problema instazionario, dato che un campo vettoriale irrotazionale si può supporre derivi da un potenziale, in questo caso della velocità:
%
	\begin{equation*}
		\uline{v} = \grad{\phi}
	\end{equation*}
%
Ma se la velocità è il gradiente di un potenziale allora:
%
	\begin{equation*}
		\dv{\uline{v}}{t} = \grad{\pdv{\phi}{t}}
	\end{equation*}
%
Quindi anche il termine irrotazionale è un gradiente, ripetendo il procedimento visto prima si ottene un quadrinomio, detto \textbf{quadrinomio di Bernoulli}:
%
	\begin{equation*}
		\grad{\left( \pdv{\phi}{t} + \frac{v^2}{t} + \frac{p}{\rho} + gz \right)} = 0
	\end{equation*}
%
La forma forte vale anche nel caso instazionario, purché si consideri il termine legato al potenziale della velocità.

%SUBSECTION
\subsection{Fluido comprimibile}
In caso di fluido comprimibile si può supporre di avere un caso in cui la densità sia variabile ma funzione solamente della pressione, $\rho = \rho(p)$.
Normalmente densità e volume specifico sarebbero funzioni di pressione e temperatura, $f = f(p, T)$, quindi sarebbe impossibile una soluzione senza tener conto dell'equazione dell'energia.
In alcuni casi però, ad esempio temperatura del fluido costante o moto isoentropico del fluido, si può eliminare la dipendenza dalla temperatura.
Quindi si può trovare:
%
	\begin{equation*}
		P = \int \frac{1}{\rho(p)} \dd{p}
	\end{equation*}
%
Questo è utile per poter scrivere:
%
	\begin{equation*}
		\begin{gathered}
			\pdv{P}{x_i} = \frac{1}{\rho} \pdv{p}{x_i}\\
			\grad{P} = \frac{1}{\rho} \grad{p}
		\end{gathered}
	\end{equation*}
%
Questo permette di riscrivere forma debole e forma forte come:
%
	\begin{equation*}
		\begin{gathered}
			\uline{v} \vdot \grad{\left( \frac{v^2}{2} + P + gz \right)} = 0\\
			 \grad{\left( \pdv{\phi}{t} + \frac{v^2}{2} + P + gz \right)} = 0
		\end{gathered}
	\end{equation*}
% 

\subsection*{Bibliografia 4.1}
\cite[Cap.\ 5.4, 5.5, 10.3, 10.4, ]{CengelCimbala}\\
\cite[Cap.\ 7.1, 7.3]{PnueliGutfinger}