%Part of/Parte di https://github.com/f-dinucci/appuntiMeccanicaFluidi/
%License/Licenza Creative Commons Attribution-ShareAlike 4.0 International (CC BY-SA 4.0) - attribution/attribuzione Francesco Di Nucci
%See also/Vedere anche https://creativecommons.org/licenses/by-sa/4.0/ and/e https://creativecommons.org/licenses/by-sa/4.0/legalcode
%
\section{Bernoulli e viscosità}

\subsection{Equazione di Bernoulli nel caso viscoso}
Nel caso viscoso il trinomio di Bernoulli non è più costante, a causa dell'energia dissipata per la viscosità del fluido.
Si definisce quindi come perdita di carico quella quantità di cui il trinomio di Bernoulli differisce da quello che si avrebbe nel caso ideale:
%
	\begin{equation*}
		\frac{v_1^2}{2} + \frac{p_1}{\rho} + g z_1 = \frac{v_2^2}{2} + \frac{p_2}{\rho} + g z_2 + \frac{\Delta p}{\rho}\\
	\end{equation*}
%
Può essere vista come una differenza di pressione $\frac{\Delta p}{\rho}$ oppure come differenza di quota.
Questa definizione è utile poiché in alcuni casi è possibile stimare la perdita di carico.

Allo stesso modo si può avere un guadagno di energia, ad esempio con una pompa per i fluidi o un ventilatore per i gas.
Questo guadagno viene detto prevalenza e anch'esso può essere misurato come differenza di pressione o di quota:
si possono avere una perdita di carico oppure un guadagno di energia (detto prevalenza, ad esempio per via di una pompa) che rispettivamente sono:
%
	\begin{equation*}
		\frac{v_1^2}{2} + \frac{p_1}{\rho} + g z_1 + g h_p = \frac{v_2^2}{2} + \frac{p_2}{\rho} + g z_2
	\end{equation*}
%

\subsection*{Bibliografia 4.4}
\cite[Cap.\ 5.4]{CengelCimbala}\\
\cite[Cap.\ 7.4]{PnueliGutfinger}