% La larghezza è impostata alla metà della pagina per sfruttare copertina fronte/retro per ottenere due colonne
% A4 210x297 mm
\documentclass[coverheight=297mm,coverwidth=105mm,spinewidth=0mm,bleedwidth=0mm,marklength=0mm]{bookcover}

% Collegamenti testuali e grafica
\usepackage{hyperref}
\usepackage{microtype}
\usepackage{tikz}

% Setup collegamenti ipertestuali e proprietà del PDF
\hypersetup{
    colorlinks=false,
    hidelinks,
    pdftitle={Appunti di Meccanica dei Fluidi},
    pdfauthor={Francesco Di Nucci},
    pdfpagemode=UseOutlines,
    pdfdisplaydoctitle =true
}

% BEGIN
\begin{document}
\begin{bookcover}

% Background Picture
\begin{bookcoverelement}{picture}{bg whole without flaps}
    ./CoverImage/CoverImage.jpg
\end{bookcoverelement}

% Grey Rectangles
% First TIKZ origin is set, then ++ tells increments starting from the origin
\begin{bookcoverelement}{tikz}{front}
    \fill[opacity=0.85,black!50] (4mm,230mm) rectangle ++(97mm,-35mm) ;
    \fill[opacity=0.85,black!50] (4mm,30mm) rectangle ++(97mm,-14mm);
\end{bookcoverelement}

% Text boxes - Position [left, down, right, up]
% Front only to use it as a column
% Top Text #1
\begin{bookcoverelement}{normal}{front}[5mm, ,5mm,70mm]
    \centering
    \huge
    Appunti di Meccanica dei Fluidi\\
    Edizione 2022-01-xx
\end{bookcoverelement}

% Top Text #2
\begin{bookcoverelement}{normal}{front}[5mm, ,5mm,90mm]
    \centering
    \Large
    \textit{Appunti non ufficiali\\
    Realizzati da Francesco Di Nucci}
\end{bookcoverelement}  

% Bottom
\begin{bookcoverelement}{normal}{front}[5mm,10mm,5mm,270mm]
    \centering
    \large
    Licenza \href{https://creativecommons.org/licenses/by-sa/4.0/}{ \texttt{CC BY-SA 4.0} }\\
    \normalsize
    \url{https://github.com/f-dinucci/appuntiMeccanicaFluidi}
\end{bookcoverelement}

\end{bookcover}
\end{document} 
% END